\documentclass{article}
\usepackage{algpseudocode,algorithm}
% Declaracoes em Português
\algrenewcommand\algorithmicend{\textbf{fim}}
\algrenewcommand\algorithmicdo{\textbf{fa\c{c}a}}
\algrenewcommand\algorithmicwhile{\textbf{enquanto}}
\algrenewcommand\algorithmicfor{\textbf{para}}
\algrenewcommand\algorithmicif{\textbf{se}}
\algrenewcommand\algorithmicthen{\textbf{ent\~ao}}
\algrenewcommand\algorithmicelse{\textbf{sen\~ao}}
\algrenewcommand\algorithmicreturn{\textbf{retorna}}
\algrenewcommand\algorithmicfunction{\textbf{fun\c{c}\~ao}}

% Rearranja os finais de cada estrutura
\algrenewtext{EndWhile}{\algorithmicend\ \algorithmicwhile}
\algrenewtext{EndFor}{\algorithmicend\ \algorithmicfor}
\algrenewtext{EndIf}{\algorithmicend\ \algorithmicif}
\algrenewtext{EndFunction}{\algorithmicend\ \algorithmicfunction}

% O comando For, a seguir, retorna 'para #1 -- #2 até #3 faça'
\algnewcommand\algorithmicto{\textbf{at\'e}}
\algrenewtext{For}[3]%
{\algorithmicfor\ #1 $\gets$ #2 \algorithmicto\ #3 \algorithmicdo}

\usepackage[utf8]{inputenc}
\usepackage{tcolorbox}
\usepackage{tabularx}
\usepackage{booktabs,tabularx,rotating}% para tabelas

\usepackage{listings}
\lstset{
  language=C++,
  basicstyle=\ttfamily\small, 
  keywordstyle=\color{blue}, 
  stringstyle=\color{verde}, 
  commentstyle=\color{red}, 
  extendedchars=true, 
  showspaces=false, 
  showstringspaces=false, 
  %numbers=left,
 % numberstyle=\tiny,
  breaklines=true, 
  backgroundcolor=\color{gray!10},
  breakautoindent=true, 
  captionpos=t,
  xleftmargin=0pt,
}

\makeatletter
\newcommand*{\rom}[1]{\expandafter\@slowromancap\romannumeral #1@}
\makeatother

\title{\centering\huge{\textsf{\textbf{Algoritmos em Grafos} \\
Trabalho Prático} \rom{1}}}
\author{\Large{\textsf{Bruno Queiroz Santos}} \\ \Large{\textsf{Gabriel Marques de Melo}}}
\date{}

% example taken from 
% http://www.guitex.org/home/images/doc/GuideGuIT/introingtikz.pdf

\usepackage {tikz}
\usetikzlibrary {positioning}
%\usepackage {xcolor}
\definecolor {processblue}{cmyk}{0.96,0,0,0}


\begin {document}

\maketitle
\vspace{60pt}
\begin {center}
\begin {tikzpicture}[-,shorten >=1pt,auto,node distance=3cm,
  thick,main node/.style={circle,fill=gray!20,draw,font=\sffamily\Large\bfseries}]
  
\node[main node] (A){$1$};
\node[main node] (B) [above right=of A] {$2$};
\node[main node] (C) [below right =of A] {$3$};
\node[main node] (D) [below right =of B] {$4$};
\node[main node] (E) [above right =of D] {$5$};
\node[main node] (F) [below right =of D] {$6$};
\node[main node] (G) [below right =of E] {$7$};

\path [every node/.style={font=\sffamily\small}]
(A) edge [above,red] node [above = 0.15cm] {30} (B)
(A) edge [above] node[below =0.15 cm] {$15$} (C)
(A) edge [above] node[above =0.15 cm] {$10$} (D)
(D) edge [above,red] node[above =0.15 cm] {$35$} (G)
(B) edge [above,red] node[above =0.15 cm] {$25$} (D)
(B) edge [above] node[above =0.15 cm] {$60$} (E)
(C) edge [above] node[above =0.15 cm] {$40$} (D)
(C) edge [above] node[below =0.15 cm] {$20$} (F)
(F) edge [above] node[below =0.15 cm] {$25$} (G)
(E) edge [above] node[above =0.15 cm] {$20$} (G);
%\path (A) edge [bend right = -15] node[below =0.15 cm] {$1/2$} (C);
%\path (A) edge [bend left =25] node[above] {$1/4$} (B);
%\path (B) edge [bend left =15] node[below =0.15 cm] {$1/2$} (A);
%\path (C) edge [bend left =15] node[below =0.15 cm] {$1/2$} (B);
%\path (B) edge [bend right = -25] node[below =0.15 cm] {$1/2$} (C);
\end{tikzpicture}

\vspace{110pt}
{\centering \Large{\textsf{UFLA \\ Novembro de 2017}}}
\thispagestyle{empty}
\end{center}



\newpage
\section{Algoritmos Implementados}

\begin{algorithm}
\caption{Problema do guia turístico}
\begin{algorithmic}[1]
\State {arquivoLista $\gets$ \textit{arquivo.txt}} 

\While {arquivoPosicao != EOF}
    \State{D\textbf{,} arquivoPosicao  $\gets$ \textit{geraCaso}(arquivoPosicao)}
    \State G $\gets$ \textit{Grafo}(D)
    \State{\textit{DijkstraModificado}(G)}
    \State{\textit{imprime} caso\textbf{,} numViagens\textbf{,} caminho }
\EndWhile
\Function{DijkstraModificado}{G}
    \While{G.flag[verticeDestino] $=$ \textit{FALSE} }
      \State{u $\gets$ \textit{MAX}(G.valor)}
      \For{i}{1}{\textit{TAM}(G.adj[u])}
        \If{G.flag[v] \textbf{\&\&} G.ma[u][v] $>=$ G.valor[u] }
            \State{G.valor[v] =  G.valor[u]}
        	\State{G.$\pi$[v] = u}
        \Else
            \State{G.valor[v] = G.ma[u][v]}
            \State{G.$\pi$[v] = u}
        \EndIf
      \EndFor
    \EndWhile
  %\If {$x < 0$}
   % \State \Return $-x$
 % \Else
  %  \State \Return $x$
  %\EndIf
\EndFunction
%\For{i}{1}{n} 
%  \State {$A[i] \gets i + 1$} \Comment{Preenche o vetor} 
%\EndFor
 % \While {$i \mei n$}
  %  \State $i \gets i + 1$
  %\EndWhile
\end{algorithmic}
\end{algorithm}

\begin{algorithm}
\caption{Problema da coleta de neve}
\begin{algorithmic}[1]
\State {arquivoLista $\gets$ \textit{arquivo.txt}} 

\end{algorithmic}
\end{algorithm}

\section{Resultados}
    Serão mostrados os resultados de \textbf{1} (uma) situação - \textit{conjunto de casos} - que cada um dos problemas \textbf{\textit{P1}} e \textbf{\textit{P2}} resolva. Para cada situação serão mostrados, respectivamente, o arquivo de entrada que contém os casos e sua tabela de resultados.
\subsection{P1}
\begin{center} \textbf{\textit{Arquivo\_de\_Entrada\_P1.txt}} \end{center}
       \begin{lstlisting} 
7 10
1 2 30
1 3 15
1 4 10
2 4 25
2 5 60
3 4 40
3 6 20
4 7 35
5 7 20
6 7 30
1 7 99
0 0

        
\end{lstlisting}

\begin{table}[!ht]
\centering
\caption{Resultados P1 \label{Resultados-P1}}
\begin{tabular}{l l l l}
   \toprule
    \textsf{Caso}    & \textsf{Caminho} & \textsf{Nº de Viagens} &  \textsf{Tempo de execução (s)}\\
    
    \midrule \midrule
    1      & 2$\rightarrow$4$\rightarrow$5$\rightarrow$6$\rightarrow$7   & 22 & 20  \\
    
    2 & 2.4 & 22 & 35 \\
    
    
    3 & 2.4 & 20 & 38  \\
    
    4 & 2.4/5 & 20/40 & 70/70
    \\[1ex]
    \bottomrule
    \end{tabular}
     
\end{table}

\subsection{P2}

\begin{center} \textbf{\textit{Arquivo\_de\_Entrada\_P2.txt}} \end{center}
       \begin{lstlisting} 
1

500 120
0 0 20 20
0 0 10000 10000
20 20 10000 10000
5000 -10000 5000 10000
5000 10000 10000 10000
12000 5000 0 0

        
\end{lstlisting}

\begin{table}[!ht]
\centering
\caption{Resultados P2 \label{Resultados-P2}}
\begin{tabular}{l l l l}
   \toprule
    \textsf{Caso}    & \textsf{Tempo da Coleta (hh:mm)} & \textsf{Tempo de execução (s)}\\
    
    \midrule \midrule
    1      & 2$\rightarrow$4$\rightarrow$5$\rightarrow$6$\rightarrow$7   & 22   \\
    
    2 & 2.4 & 22 \\
    
    
    3 & 2.4 & 20 \\
    
    4 & 2.4/5 & 20/40 
    \\[1ex]
    \bottomrule
    \end{tabular}
     
\end{table}


\end{document}